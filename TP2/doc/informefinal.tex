\documentclass[a4paper,10pt]{article}

\usepackage[utf8]{inputenc}
\usepackage{t1enc}

\usepackage[utf8]{inputenc}
\usepackage{t1enc}
\usepackage[spanish]{babel}
\usepackage[pdftex,usenames,dvipsnames]{color}
\usepackage[pdftex]{graphicx}
\usepackage{enumerate}
\usepackage{amsmath}
\usepackage{amsfonts}
\usepackage{amssymb}
\usepackage[table]{xcolor}
\usepackage[small,bf]{caption}
\usepackage{float}
\usepackage{subfig}
\usepackage{listings}
\usepackage{bm}
\usepackage{times}

\begin{document}
\setcounter{secnumdepth}{5}
\setcounter{tocdepth}{5}

\renewcommand{\lstlistingname}{C\'odigo Fuente}
\lstloadlanguages{Octave} 
\lstdefinelanguage{MyOctave}[]{Octave}{
        deletekeywords={beta,det},
        morekeywords={repmat}
} 
\lstset{
        language=MyOctave,
        stringstyle=\ttfamily,
        showstringspaces = false,
        basicstyle=\footnotesize\ttfamily,
        commentstyle=\color{gray},
        keywordstyle=\bfseries,
        numbers=left,
        numberstyle=\ttfamily\footnotesize,
        stepnumber=1,                   
        framexleftmargin=0.20cm,
        numbersep=0.37cm,              
        backgroundcolor=\color{white},
        showspaces=false,
        showtabs=false,
        frame=l,
        tabsize=4,
        captionpos=b,               
        breaklines=true,             
        breakatwhitespace=false,      
        mathescape=true
}

%%%%%%%%%%%%%%%%%%%%%%%%%%%%%%%%%%
%%%%%%%% begin TITLE PAGE %%%%%%%%
%%%%%%%%%%%%%%%%%%%%%%%%%%%%%%%%%%
\begin{titlepage}
        \vfill
        \thispagestyle{empty}
        \begin{center}
                \includegraphics{./images/itba_logo.png}
                \vfill
                \Huge{Sistemas de Inteligencia Artificial}\\
                \vspace{1cm}
                \Huge{Redes Neuronales}\\
                \vspace{1cm}
                \Huge{Trabajo Pr\'actico Especial 2}\\
        \end{center}
        \vfill
        \large{
        \begin{tabular}{lcr}
                Civile, Juan Pablo && 50453\\
                Ordano, Esteban && 50753\\
                Crespo, Alvaro && 50758 \\
        \end{tabular}
}
        \vspace{2cm}
        \begin{center}
                \large{24 de abril de 2013}\\
        \end{center}
\end{titlepage}
\newpage

\setcounter{page}{1}

% \tableofcontents
% \newpage

\section{Definición del problema}

El problema consiste en la implementación de una red neuronal multicapa que realice la predicción de series temporales. Para ello se supone que $x(t)$ (el valor de la 
serie en el paso temporal $t$) es alguna función desconocida de los valores de la serie en pasos anteriores, es decir, $x(t) = f (x(t - 1), x(t - 2), x(t - 3), \dots)$. 
La red neuronal deberá poder aproximar a la función $f$ desconocida. A priori no se conoce cuantos pasos temporales previos hay que considerar como argumento 
de la función $f$. Se utilizaron siempre menos de cuatro.

\section{Implementacion}

\subsection{Mejoras}

\subsubsection{\eta adaptativo}

\subsubsection{Momentum}

\subsubsection{Ruido}

\section{Resultados}

\section{Conclusiones}

\section*{Anexo: Resultados}

\begin{table}[H]
    \begin{tabular}{c|r|r|r}
        Arquitectura & Error & Std. Dev. & Error Max  \\
        \hline
        5x3 & 0.041884 & 0.067486 & 0.25286 \\
        5x4 & 0.019276 & 0.029188 & 0.10784 \\
        6x3 & 0.030846 & 0.047927 & 0.15913 \\
        6x4 & 0.025126 & 0.033176 & 0.13532 \\
        7x3 & 0.026747 & 0.033905 & 0.13142 \\
        6x5 & 0.018454 & 0.027995 & 0.10857 \\
        7x4 & 0.018442 & 0.027286 & 0.098194 \\
        8x3 & 0.039789 & 0.050507 & 0.1554 \\
        7x5 & 0.018988 & 0.028982 & 0.1068 \\
        8x4 & 0.022146 & 0.030715 & 0.10197 \\
        9x3 & 0.036945 & 0.047534 & 0.14866 \\
        7x6 & 0.018294 & 0.026754 & 0.11407 \\
        8x5 & 0.018705 & 0.029107 & 0.099821 \\
        9x4 & 0.019588 & 0.028886 & 0.10269 \\
        8x6 & 0.019858 & 0.028843 & 0.10546 \\
        9x5 & 0.017007 & 0.026745 & 0.10267 \\
        8x7 & 0.017853 & 0.026049 & 0.10511 \\
        9x6 & 0.018605 & 0.025973 & 0.093161 \\
        9x7 & 0.019796 & 0.027959 & 0.092556 \\
        9x8 & 0.01605 & 0.025738 & 0.11786 \\
        10x7 & 0.024545 & 0.034505 & 0.10629 \\
        10x8 & 0.02795 & 0.034333 & 0.10529 \\
        11x7 & 0.031485 & 0.037798 & 0.095195 \\
        10x9 & 0.017352 & 0.026223 & 0.12413 \\
        11x8 & 0.022803 & 0.030054 & 0.092118 \\
        11x9 & 0.029583 & 0.039647 & 0.11379 \\
    \end{tabular}

    \caption{Detalle de los valores obtenidos para distintas arquitecturas utilizando la funcion $\tanh$ y usando los parametros optimos}
    \label{tab:arquitectura_full}
 
\end{table}


\end{document}
